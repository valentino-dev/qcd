\documentclass[10pt]{article}
\usepackage[a4paper, left=1.5cm, right=1.5cm, top=3.5cm]{geometry}
\usepackage[]{graphicx}
\usepackage{multicol}
\usepackage{amssymb}
\usepackage{titlesec}
\usepackage{wrapfig}
\usepackage{blindtext}
\usepackage{lipsum}
\usepackage{caption}
\usepackage{listings}
\usepackage{fancyhdr}
\usepackage{nopageno}
\usepackage{authblk}
\usepackage{amsmath} % tons of math stuff
\usepackage{mathtools} % e.g. alignment within matrix
\usepackage{bm} % provides shorthand for bold in math mode
\usepackage{esdiff} % provides derivative commands
\usepackage{xcolor}
\usepackage{csquotes} % e.g. provides \enquote
\fancyhf[]{}

% own fig. env. for multicols
\newenvironment{Figure}
  {\par\medskip\noindent\minipage{\linewidth}}
  {\endminipage\par\medskip}

\begin{titlepage}
    \title{Lattice QCD}
    \author[1]{Angelo V. Brade\thanks{s72abrad@uni-bonn.de}}
    \affil[1]{Rheinische Friedrich-Wilhelms-Universität Bonn}
    \date{\today}
\end{titlepage}

\begin{document}
\pagenumbering{gobble}
\maketitle
\newpage

\tableofcontents
\newpage

\pagenumbering{arabic}

\pagestyle{fancy}
\fancyhead[R]{\thepage}
\fancyhead[L]{\leftmark}



\section{p2gg}
\subsection{Lower bound}
\begin{Figure}
  \centering\resizebox{\textwidth}{!}{\input{LowerBound}}
  \captionof{figure}{
    Lower bound for stable correlations.
  }
  \label{fig:1.1}
\end{Figure}

%Es lässt sich erkennen, dass bei \(t_\text{lower}=5\) der Fehler \(\Delta E\) am geringsten ist. Somit werden fortan all vorherien abgeschnitten.

\subsection{Upper bound}
\begin{Figure}
  \centering\resizebox{\textwidth}{!}{\input{UpperBound}}
  \captionof{figure}{
    Upper bound for stable correlations.
  }
  \label{fig:1.2}

\subsection{Results}
\end{Figure}
\begin{Figure}
  \centering\resizebox{\textwidth}{!}{\input{CorrelationFunction.tex}}
  \captionof{figure}{
    Lower bound for stable correlations.
  }
  \label{fig:1.3}
\end{Figure}

Wir finden die in Tablle \ref{Tab:1.1} aufgeführten Werte, wobei die Correlatoren von \(t_\text{lower}=16\) und \(t_\text{upper}=80\) ausgewertet wurden, mit \(\chi^2=0.2657\), \(X_0:=\) X ohne Bootstrapping, \(X_\text{boot.}:=\) X mit Bootstrapping, \(\frac{X_0 - X_\text{boot.}}{X_0}:=\) der normierten relativen Abweichung des Bootstrapps zu dem originalem Wert und \(\frac{\sigma_{X_\text{boot.}}}{X_0}:=\) dem normierten Fehler des Bootstrapps.

\begin{center}
\begin{tabular}{ccccccc}
  \(X\)&\(X_0\)&\(\sigma X_0\)&\(X_\text{boot.}\)&\(\sigma X_\text{boot.}\)&\(\frac{X_0 - X_\text{boot.}}{X_0}\)&\(\frac{\sigma_{X_\text{boot.}}}{X_0}\)\\
  \hline
  C & 3.35e-03 & 1.77e-05 & 3.34e-03 & 2.96e-05 & 4.83e-03 & 8.83e-03\\
  E & 4.73e-02 & 3.10e-05 & 4.73e-02 & 4.79e-05 & -4.71e-05 & 1.01e-03
\end{tabular}
\captionof{table}{Ergebnisse zur p2gg Auswertung.}
\label{Tab:1.1}
\end{center}


\end{document}
