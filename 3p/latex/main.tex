\documentclass[10pt]{article}
\usepackage[a4paper, left=1.5cm, right=1.5cm, top=3.5cm]{geometry}
\usepackage[]{graphicx}
\usepackage{multicol}
\usepackage{amssymb}
\usepackage{inputenc}
\usepackage{breqn}
\usepackage{titlesec}
\usepackage{wrapfig}
\usepackage{blindtext}
\usepackage{lipsum}
\usepackage{caption}
\usepackage{listings}
\usepackage{fancyhdr}
\usepackage{nopageno}
\usepackage{authblk}
\usepackage{amsmath}
\usepackage{mathtools}
\usepackage{bm}
\usepackage[ISO]{diffcoeff}
\usepackage{xcolor}
\usepackage{csquotes}
\usepackage{siunitx}
\usepackage{circuitikz}
\fancyhf[]{}

% own fig. env. for multicols
\newenvironment{Figure}
  {\par\medskip\noindent\minipage{\linewidth}}
  {\endminipage\par\medskip}

\begin{titlepage}
    \title{Lattice QCD}
    \author[1]{Angelo V. Brade\thanks{s72abrad@uni-bonn.de}}
    \affil[1]{Rheinische Friedrich-Wilhelms-Universität Bonn}
    \date{\today}
\end{titlepage}

\begin{document}
\pagenumbering{gobble}
\maketitle

\pagenumbering{arabic}

\pagestyle{fancy}
\fancyhead[R]{\thepage}
\fancyhead[L]{\leftmark}

\begin{multicols}{2}
\section{3 Punkt Funktionen}
3-Punkt Funktionen aus\\ \texttt{p2gg_local_neutral_light.p-lvc-lvc.fl1.} \\ \texttt{qx0_qy0_qz0.gseq_4.tseq_15.px0_py-2_pz2.h5}.
\subsection{}
\begin{Figure}
  \centering\resizebox{\textwidth}{!}{% GNUPLOT: LaTeX picture with Postscript
\begingroup
  \makeatletter
  \providecommand\color[2][]{%
    \GenericError{(gnuplot) \space\space\space\@spaces}{%
      Package color not loaded in conjunction with
      terminal option `colourtext'%
    }{See the gnuplot documentation for explanation.%
    }{Either use 'blacktext' in gnuplot or load the package
      color.sty in LaTeX.}%
    \renewcommand\color[2][]{}%
  }%
  \providecommand\includegraphics[2][]{%
    \GenericError{(gnuplot) \space\space\space\@spaces}{%
      Package graphicx or graphics not loaded%
    }{See the gnuplot documentation for explanation.%
    }{The gnuplot epslatex terminal needs graphicx.sty or graphics.sty.}%
    \renewcommand\includegraphics[2][]{}%
  }%
  \providecommand\rotatebox[2]{#2}%
  \@ifundefined{ifGPcolor}{%
    \newif\ifGPcolor
    \GPcolortrue
  }{}%
  \@ifundefined{ifGPblacktext}{%
    \newif\ifGPblacktext
    \GPblacktexttrue
  }{}%
  % define a \g@addto@macro without @ in the name:
  \let\gplgaddtomacro\g@addto@macro
  % define empty templates for all commands taking text:
  \gdef\gplbacktext{}%
  \gdef\gplfronttext{}%
  \makeatother
  \ifGPblacktext
    % no textcolor at all
    \def\colorrgb#1{}%
    \def\colorgray#1{}%
  \else
    % gray or color?
    \ifGPcolor
      \def\colorrgb#1{\color[rgb]{#1}}%
      \def\colorgray#1{\color[gray]{#1}}%
      \expandafter\def\csname LTw\endcsname{\color{white}}%
      \expandafter\def\csname LTb\endcsname{\color{black}}%
      \expandafter\def\csname LTa\endcsname{\color{black}}%
      \expandafter\def\csname LT0\endcsname{\color[rgb]{1,0,0}}%
      \expandafter\def\csname LT1\endcsname{\color[rgb]{0,1,0}}%
      \expandafter\def\csname LT2\endcsname{\color[rgb]{0,0,1}}%
      \expandafter\def\csname LT3\endcsname{\color[rgb]{1,0,1}}%
      \expandafter\def\csname LT4\endcsname{\color[rgb]{0,1,1}}%
      \expandafter\def\csname LT5\endcsname{\color[rgb]{1,1,0}}%
      \expandafter\def\csname LT6\endcsname{\color[rgb]{0,0,0}}%
      \expandafter\def\csname LT7\endcsname{\color[rgb]{1,0.3,0}}%
      \expandafter\def\csname LT8\endcsname{\color[rgb]{0.5,0.5,0.5}}%
    \else
      % gray
      \def\colorrgb#1{\color{black}}%
      \def\colorgray#1{\color[gray]{#1}}%
      \expandafter\def\csname LTw\endcsname{\color{white}}%
      \expandafter\def\csname LTb\endcsname{\color{black}}%
      \expandafter\def\csname LTa\endcsname{\color{black}}%
      \expandafter\def\csname LT0\endcsname{\color{black}}%
      \expandafter\def\csname LT1\endcsname{\color{black}}%
      \expandafter\def\csname LT2\endcsname{\color{black}}%
      \expandafter\def\csname LT3\endcsname{\color{black}}%
      \expandafter\def\csname LT4\endcsname{\color{black}}%
      \expandafter\def\csname LT5\endcsname{\color{black}}%
      \expandafter\def\csname LT6\endcsname{\color{black}}%
      \expandafter\def\csname LT7\endcsname{\color{black}}%
      \expandafter\def\csname LT8\endcsname{\color{black}}%
    \fi
  \fi
    \setlength{\unitlength}{0.0500bp}%
    \ifx\gptboxheight\undefined%
      \newlength{\gptboxheight}%
      \newlength{\gptboxwidth}%
      \newsavebox{\gptboxtext}%
    \fi%
    \setlength{\fboxrule}{0.5pt}%
    \setlength{\fboxsep}{1pt}%
    \definecolor{tbcol}{rgb}{1,1,1}%
\begin{picture}(7200.00,4320.00)%
    \gplgaddtomacro\gplbacktext{%
      \csname LTb\endcsname%%
      \put(1025,619){\makebox(0,0)[r]{\strut{}$-6\times10^{-5}$}}%
      \csname LTb\endcsname%%
      \put(1025,1062){\makebox(0,0)[r]{\strut{}$-4\times10^{-5}$}}%
      \csname LTb\endcsname%%
      \put(1025,1505){\makebox(0,0)[r]{\strut{}$-2\times10^{-5}$}}%
      \csname LTb\endcsname%%
      \put(1025,1947){\makebox(0,0)[r]{\strut{}$0$}}%
      \csname LTb\endcsname%%
      \put(1025,2390){\makebox(0,0)[r]{\strut{}$2\times10^{-5}$}}%
      \csname LTb\endcsname%%
      \put(1025,2833){\makebox(0,0)[r]{\strut{}$4\times10^{-5}$}}%
      \csname LTb\endcsname%%
      \put(1025,3276){\makebox(0,0)[r]{\strut{}$6\times10^{-5}$}}%
      \csname LTb\endcsname%%
      \put(1025,3719){\makebox(0,0)[r]{\strut{}$8\times10^{-5}$}}%
      \csname LTb\endcsname%%
      \put(1123,425){\makebox(0,0){\strut{}$0$}}%
      \csname LTb\endcsname%%
      \put(1843,425){\makebox(0,0){\strut{}$20$}}%
      \csname LTb\endcsname%%
      \put(2564,425){\makebox(0,0){\strut{}$40$}}%
      \csname LTb\endcsname%%
      \put(3284,425){\makebox(0,0){\strut{}$60$}}%
      \csname LTb\endcsname%%
      \put(4004,425){\makebox(0,0){\strut{}$80$}}%
      \csname LTb\endcsname%%
      \put(4725,425){\makebox(0,0){\strut{}$100$}}%
      \csname LTb\endcsname%%
      \put(5445,425){\makebox(0,0){\strut{}$120$}}%
      \csname LTb\endcsname%%
      \put(6165,425){\makebox(0,0){\strut{}$140$}}%
      \csname LTb\endcsname%%
      \put(6886,425){\makebox(0,0){\strut{}$160$}}%
    }%
    \gplgaddtomacro\gplfronttext{%
      \csname LTb\endcsname%%
      \put(6123,3545){\makebox(0,0)[r]{\strut{}$\nu=1$}}%
      \csname LTb\endcsname%%
      \put(6123,3351){\makebox(0,0)[r]{\strut{}$\nu=2$}}%
      \csname LTb\endcsname%%
      \put(6123,3158){\makebox(0,0)[r]{\strut{}$\nu=3$}}%
      \csname LTb\endcsname%%
      \put(6123,2964){\makebox(0,0)[r]{\strut{}$\nu=4$}}%
      \csname LTb\endcsname%%
      \put(170,2169){\rotatebox{-270.00}{\makebox(0,0){\strut{}$C_{\mu\nu}^{(3)}(t_c-t_i)$}}}%
      \csname LTb\endcsname%%
      \put(4004,135){\makebox(0,0){\strut{}$t_c-t_i$}}%
      \csname LTb\endcsname%%
      \put(4004,4009){\makebox(0,0){\strut{}Correlation function with $\mu=1$}}%
    }%
    \gplbacktext
    \put(0,0){\includegraphics[width={360.00bp},height={216.00bp}]{p3f_mu1}}%
    \gplfronttext
  \end{picture}%
\endgroup
}
  \captionof{figure}{
    3-Punkt Funktion mit $\mu=1$
  }
  \label{fig:1.1}
\end{Figure}
\begin{Figure}
  \centering\resizebox{\textwidth}{!}{\input{p3f_mu2}}
  \captionof{figure}{
    3-Punkt Funktion mit $\mu=2$
  }
  \label{fig:1.2}
\end{Figure}
\begin{Figure}
  \centering\resizebox{\textwidth}{!}{\input{p3f_mu3}}
  \captionof{figure}{
    3-Punkt Funktion mit $\mu=3$
  }
  \label{fig:1.3}
\end{Figure}
\begin{Figure}
  \centering\resizebox{\textwidth}{!}{\input{p3f_mu4}}
  \captionof{figure}{
    3-Punkt Funktion mit $\mu=4$
  }
  \label{fig:1.4}
\end{Figure}
Hierbei wurde der shift $e^{-\mathrm{i}(\vec{p}+\vec{q})\vec{r}}$ mit $\vec{q}=\vec{0}$ als $\cos{(-\vec{p}\cdot\vec{r})}$ interpretiert, da wir nur den Realteil betrachten.


\end{multicols}
\end{document}
